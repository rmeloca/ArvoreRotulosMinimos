\documentclass[12pt]{article}

\usepackage{sbc-template}
\usepackage{graphicx,url}
\usepackage[brazil]{babel}
\usepackage[latin1]{inputenc}
\usepackage{lscape}
\usepackage{geometry}
\usepackage{float}

\sloppy

\title{Algoritmo A* para resolução da Árvore Geradora de Rótulos Mínimos}

\author{Marco Cezar Moreira de Mattos\inst{1}, Rômulo Manciola Meloca\inst{1}}

\address{DACOM -- Universidade Tecnológica Federal do Paraná (UTFPR)\\
  Caixa Postal 271 -- 87301-899 -- Campo Mourão -- PR -- Brazil
  \email{\{marco.cmm,rmeloca\}@gmail.com}
}

\begin{document}

\maketitle
     
\begin{resumo} 
  Este relatório apresenta o emprego do algoritmo A* para a resolução da Árvore Geradora de Rótulos Mínimos, desenvolvimento de um software servidor de arquivos, desde a sua concepção, apresentando a solução projetada até a fase de implementação do mesmo. Lê-se neste decisões de projeto e obstáculos contornados.
\end{resumo}

\section{Introdução}\label{sec:introducao}

\section{O problema}\label{sec:problema}

\section{Trabalhos Relacionados}\label{sec:trabalhosRelacionados}

\section{Solução Proposta}\label{sec:solucao}

	Implementou-se a solução na linguagem de programação Java, aproveitando códigos previamente implementados.

\subsection{Busca A*}\label{sec:aestrela}

	Sendo o PAGRM um problema NP-Completo em sua solução ótima 

\subsection{Diagramação}\label{sec:diagramacao}

\newgeometry{left=0cm,bottom=0cm,right=0cm,top=0cm}
\begin{landscape}
\centering
\begin{figure}[p]
\includegraphics[width=1.4\textwidth]{ClassDiagram.png}
\caption{Diagrama de Classes}
\label{fig:classDiagram}
\end{figure}
\end{landscape}
\restoregeometry

\section{Resultados}\label{sec:resultados}

\subsection{Análise}\label{sec:analise}

\section{Considerações Finais}\label{sec:conclusao}

\section{Referências}\label{sec:referencia}

\end{document}